% DOCUMENTATION
% https://fr.wikibooks.org/wiki/LaTeX

\documentclass[a4paper, 11pt]{book}
\usepackage[utf8]{inputenc}
\usepackage{graphicx}
% \usepackage[frenchb]{babel}
\begin{document}

\title{Une histoire de l'informatique}
\author{Alexandre Bouvier}
\date{}

\mainmatter

\maketitle

\chapter*{ Introduction }

\section*{ La Logique à l'origine des ordinateurs }

\paragraph{
  La recherche de la vérité a toujours été à la base des Sciences. Et dans une démonstration scientifique, la preuve logique est la voie royale d'accès à la vérité. Apparu chez les philosophes grecs, elle a été enrichi par l'anglais, Georges Boole qui lui a donné ses lettres de noblesse. De 1844 à 1854, il crée une algèbre binaire, n'acceptant que deux valeurs numériques : 0 et 1. on appellera cette logique booléenne. Il veut affranchir la logique de la réduction aux mathématiques à laquelle elle était étroitement lié et veut en faire une discipline à part entière en mettant sur pied les notions de quantificateur (ET, OU, QUELQUE SOIT, IL EXISTE, IMPLICATION, EQUIVALENCE). Par la suite, les logiciens voulaient que le nouveau langage logique donne aux mathématiques de solides fondements (nottamment via la Théorie des Ensembles). La logique, pour Hilbert, est une image de la plus haute forme de vérité. Selon lui "les Mathématiques ne doivent plus rien admettre  qui soit intuitivement évident" comme par exemple deux droites parallèles ne se croisent jamais.
}

\paragraph{
  Plusieurs paradoxe mettent en avant l'illogisme du langage courant et les mathématiciens ne veulent pas tomber dans les mêmes travers. Mais cette recherche de la vérité en Mathématique basé sur la logique butte sur plusieurs problèmes. En 1900 à l'occasion d'un congrès international de mathématiciens tenu à Paris, David Hilbert, le mathématicien allemand, propose sa fameuse liste des 23 problèmes que les Mathématiques moderne n'avaient pas résolus. Même au XXIe siècle, elle est considérée comme étant la compilation de problèmes ayant eu le plus d'influence en mathématiques.
}

\paragraph{
  Malheureusement dans les sciences en générale et dans les Mathématiques en particulier cette recherche de la vérité absolue n'a jamais abouti. Elle a même rendu fou de nombreux savants. En 1920 Godel siffle la fin de la récréation avec son théorème d'incomplétude.
}

\section*{ Théorème d'incomplétude de Godel }

\paragraph{
  Ce théorème affirme qu'en Mathématiques, il existerait toujours des vérités qui sont indémontrables. Von Neumann dit que c'est terminé en entendant le théorème d'incomplétude de Gödel. Alan Turing répond "OK on peut pas tout prouver! Donc voyons ce qu'on peut prouver".
}

\section*{ Alan Turing }

\paragraph{
  Alan Turing est alors tout jeune chercheur quand Godel expose son théorème d'incomplétude. La logique qui est la voie royale des sciences ne séduit plus. La mode est à la résolution des problèmes d'Hilbert. Le cambrdigien s'arrete sur le 11ème. En 1936, la publication d'un article de logique mathématique On computable Numbers, with an Application to the Entscheidungsproblem Référence de la publication constitue avec d'autres recherches fondamentales un cadre théorique qui intéressera plus tard les fondateurs de la "science informatique". La machine de Turing est une abstraction modélisant un "être calculant" pour démontrer une proposition de logique pure. À l'époque n'a rien à voir avec un projet de machine.
}

\paragraph{
  Tous les ordinateurs et microprocesseurs d'aujourd'hui sont encore basés sur le principe énoncé par Turing. Son aspect élémentaire et basique sert également à déterminer si un calcul peut être réalisé sous forme automatique (algorithmique).
}

\chapter{ Les Racines }

\paragraph{
  Il y'a eu une première révolution industrielle ou on avait inventé le moteur. Et ce moteur a remplacé le muscle humain c'est à dire qu'on a remplacé des ouvriers qui utilisait leur muscles par des moteurs. Parce que c'est moins chère. Ca a mécanisé le travail agricole et le travail manufacturier. Mais ça l'a pas totalement remplacé.
}

\paragraph{
  L'invention de l'ordinateur c'est la révolution industrielle de la dématérialisation du cerveau humain. Cad que les comptables qui faisait des calculs avec le grand livre. Maintenant le grand livre est stocke dans un ordinateur et le calcul est lui aussi fait par l'ordinateur. Et les comptables peuvent se concentrer sur l'optimisation fiscale. On pourrait imaginer que plus tard le *machine learning* remplace l'optimisation fiscale mais ce n'est pas un roman d'anticipation. Y'a plein de job qui se font automatisé. L'informatique c'est la machine outil pour faire remplacer les métiers de service.
}

\paragraph{
  Basile Bouchon en 1725 est le premier à utilisé une programmation à ruban perforé pour les métiers à tisser.
}

\paragraph{
  Tout commence avec le recensement aux États-Unis en 1890. Un concours est lancé pour trouver une méthode rapide pour compter les Américains. Herman Hollerith propose un tabulateur. Le comptage est terminé en six semaines avec cinq millions de dollars d'économie. La machine sort: 62 927 766 ressortissants. Le calculateur analogique qui compte le nombre de trous percés dans chaque carte.
}

\paragraph{
  Il s'agissait d'une machine à cartes perforées. Les trous déclenchent la progression des compteurs. Un prémisse de l'ordinateur. Babbage avait eu la même idée avant lui auparavant mais Hollerith en a fait un business: International Business Machines Corporation (IBM). Et il dépose un brevet de la machine à cartes perforées.
}

\paragraph{
  En 1939, les nazis souhaitent aussi faire un recensement de la population et font appel à Dehomag, une succursale allemande d'IBM.
}


\chapter{ Première génération d'ordinateurs ou machine turo-compatible (1936-1956) }

\section{
  Ordinateurs analogiques.
}

\paragraph{
  Avant la seconde guerre mondiale, les ordinateurs analogiques étaient tous mécaniques ou électriques. Ces ordinateurs utilisaient des quantités physiques, telles que la tension,le courant ou la vitesse de rotation des axes pour représenter les nombres. À partir de la seconde guerre mondiale, le calcul numérique remplaça le calcul analogique.
}

\paragraph{
  Les premières machine de Turing ou plutôt machine Turing-compatible servent à calculer des trajectoires de missiles. Cela permettait de calculer des trajectoires de tirs beaucoup plus rapidement qu'un être humain le ferait. Et en temps de guerre, les secondes sont précieuses. En 1938, Konrad Zuse commença la construction des premières séries-Z, des calculateurs électromécaniques comportant une mémoire et une programmation limitée. La Wehrmacht utilisa ces systèmes pour des missiles guidés.
}

\paragraph{
  Durant la même période., en 1938, John Vincent Atanasoff et Clifford E. Berry, de l'Université de l'État de l'Iowa, développèrent l'ordinateur Atanasoff-Berry un additionneur à 16 bits. Cette machine avait pour but de résoudre des systèmes d'équation linéaires. La mémoire était stcokée à l'aide de condensateurs fixés à un tambour rotatif.
}

\paragraph{
  Les trois idées propres aux ordinateurs d'après guerre sont:
  - l'utilisation du sytème binaire (plus fiable et plus simple à mettre au point que le système décimal)
  - la séparation entre le calcul et la mémoire (Von Neumann ?)
  - l'utilisation de composants électroniques plutôt que des éléments mécaniques pour réaliser les calculs
}


\chapter{ WWII }

\paragraph{
  L'ENIAC a été commandé en 1943 par l'armée américaine afin d'effectuer les calculs de balistique, c'est à dire toujours des calculs de trajectoire de missiles. Ce sont des calculs longs et fastidieux qui sont fait en moins d'une seconde par l'ordinateur au lieu des 10 minutes par un humain. La machine tombait souvent en panne mais elle était plutôt fiables pour l'époque. L'ENIAC pesait plus de 30 tonnes  et occupait 167 m2. Elle était composée de 20 calculateurs fonctionnant en parallèle et pouvait effectuer 100 000 additions ou 357 multiplications par seconde.
}

\paragraph{
  À partir de 1948 apparurent les premières machines à architecture de von Neumann. Contrairement aux machines précédentes, les logiciels et programmes ne sont pas stockés dans le même espace mémoire que les données. Tous les ordinateurs actuels sont de type von Neumann.
}

\paragraph{
  L'EDVAC : En 1945 Von Neumann fait un algorithme de tri pour l'EDVAC.
}


\chapter{ 1945 }

\paragraph{
  En 1952, IBM produit son premier ordinateur scientifique, l'IBM 701. Il est destiné à la défense américaine. Il effectuait 16 000 additions ou 2 200 multiplications par seconde. L'objectif était de calculer des trajectoires de missiles.
}

\paragraph{
  En 1953, IBM lance l'IBM 650, ordinateur commercial (ou de gestion). Il se présentait en 2 modules de 2,5 m3, l'un de 900kg contenant l'ordinateur, l'autre de 1350kg contenant son alimentation électrique.
}

\paragraph{
  Le premier langage de programmation universel de haut niveau à être implémenté, le Fortran (Formula Translator), fut aussi développé par IBM à cette période.
}

\paragraph{
  En 1958, IBM rate l'opportunité de racheter une jeune société en expansion qui avait mis au point une nouvelle technologie d'impressions appelée la xerographie. Deux ans plus tard va naître Xerox.
}

\paragraph{
  En 1960, la Compagnie des Machines Bull (France) sort le Gamma 60. C'est le premier ordinateur multitâches dans le monde et l'un des premiers à comporter plusieurs processeurs (voir multiprocesseur).
}

\paragraph{
  La même année, Digital Equipment Corporation (DEC) lança le PDP-1 (Programmed Data Processor). Le PDP-1 était le premier ordinateur interactif et a lancé le concept de mini-ordinateur. Il effectuait 100 000 opérations par seconde. Vendu pour 120 000 dollars.
}


\chapter{ 1950 }

\paragraph{
  Avant l'apparition des ordinateurs personnels il n'y avait que des grosses machines inutilisable pour le communs des mortels. On les appelait alors les mainframe computer. Les utilisateurs étaient surtout des chercheurs en intelligence artificiel, mathématiciens et quelques passionnées. Un des premiers ordinateur était le PDP 10. Dans les années 70, c'est sur cette ordinateur (mainframe computer) que  Richard Stallman (initiateur du mouvement du logiciel libre), Bill Gates et Paul Allen (les deux fondateurs de Microsoft) font leur premier pas en informatique. Steve Wozniak va lui commencé sur un ENIAC et un PDP-8 de Digital Equipment. Le constructeur principal des ces machines était IBM et sa machine le IBM 7094.
}

\paragraph{
  Le moyen d'intéragir avec ces grosses machines était,comme aujourd'hui, le système d'exploitation (Operating system ou OS). Et le plus populaires d'entre eux était UNIX. C'est un système d'exploitation (OS) créé en 1969 par Kenneth Thompson alors ingénieur au laboratoire Dell detenu par ATT. Il avait ensuite continué à être développé par les universités américaines. Il était compatible sur de nombreuses machines alors qu'à l'époque chaque ordinateur avait son propre système d'exploitation. Sur Unix, pour faire vite, vous donniez des instructions à un interprétateur de ligne de commande, un shell comme ci dessous. Le système d'exploitation se chargait de les exécuter. D'une manière générale, la quasi-totalité des PC ou mobile les plus courants (à l'exception des Windows) sont basés sur le noyau de Unix. Y compris ceux commercialisés par Apple. La taille des processeurs est immenses à l'époque, de plusieurs mètres cubes. Les ordniateurs tiennent dans plusieurs salles.
}

\paragraph{
  Au début des années 1960, les plus grosses sociétés d'informatiques sont IBM, Xerox, Dell, Commodore. Ce ne sont ques des constructeurs de machines physiques, du hardware. Pour elles, les OS et logiciels ne sont que la cerise sur la gateau. Ce sont les utilisateurs qui doivent écrire les logiciels. Etant donné que la plupart sont des chercheurs en Inteligence Artificielle, mathématiciens et quelques passionnées cela ne pose pas de problèmes.
}

\paragraph{
  Les circuits intégrés : Les premiers ordinateurs utilisant les circuits intégrés sont apparus en 1963. L'un des premiers usage a été dans les systèmes embarqués, notamment par la NASA dans l'ordinateur de guidage d'Apollo et par les militaires dans le missile balistique intercontinental LGM-30. Le circuit intégré permet le développement d'ordinateurs plus compacts.
}

\paragraph{
  En 1965, DEC lance le PDP-8.
}

% \paragraph{
%   Les générations de langages:
%   1 codage machine direct en binaire
%   2 langage assembleur
%   3 langages évolués: Fortran, COBOL, Simula, APL etc.)
%   4 langages structurés (Pascal, C++) et langage Objets
% }

\paragraph{
  ARPANET (Source Wiki): ARPANET ou Arpanet (acronyme anglais de " Advanced Research Projects Agency Network ", souvent typographié ARPAnet) est le premier réseau à transfert de paquets développé aux États-Unis par la DARPA. Le projet fut lancé en 19662, mais ARPANET ne vit le jour qu'en 1969. Sa première démonstration officielle date d'octobre 1972.
}

\paragraph{
  Arpanet est créé afin d'unifier les techniques de connexion pour qu'un terminal informatique se raccorde à distance à des ordinateurs de constructeurs différents.
}

\paragraph{
  Le concept de commutation de paquets (packet switching), qui deviendra la base du transfert de données sur Internet, était alors balbutiant dans la communication des réseaux informatiques. Les communications étaient jusqu'alors basées sur la communication par circuits électroniques, telle que celle utilisée par le réseau de téléphone, où un circuit dédié est activé lors de la communication avec un poste du réseau.
}

\paragraph{
  Les ordinateurs utilisés étaient principalement des ordinateurs commerciaux de 3e génération construits par Digital Equipment Corporation (DEC), International Business Machines (IBM) ou Scientific Data Systems. Peut-être comprenaient-ils encore des Univac à tubes électroniques, technologie certes désuète en 1969 (où on abandonnait déjà les ordinateurs de deuxième génération transistorisés pour d'autres à circuits intégrés comme l'IBM 1130), mais c'est précisément pour cela que ces ordinateurs étaient libres pour un usage expérimental, les autres étant saturés de travaux3.
}

\paragraph{
  ARPANET a été écrit par le monde universitaire et non militaire, ce qui a probablement influencé l'Internet que l'on connait aujourd'hui4.
}

\paragraph{
  Le 29 Octobre 1969 le premier message était envoyé avec la machine ci-dessous sur un réseau appelé Arpanet entre les universités de Stanford et l'université de Californie à Los Angeles. C'est la naissance d'internet.
}


\chapter{ 1960 }

\paragraph{
  Le processeur est le cerveau de l'ordinateur. On l'appelle aussi CPU, pour Central Processing Unit, qui se traduit par Unité Centrale de Traitement. C'est lui qui permet de manipuler des informations codées sous forme binaire, c'est à dire de Un et de Zéro. Et d'exécuter des instructions à partir de ses informations. Mais il ne permet pas de stocker des informations. Il est juste un éxécutant.
}

\paragraph{
  Le premier microprocesseur (Intel 4004) a été inventé en 1971. Il s'agissait d'une unité de calcul de 4 bits, cadencé à 108 kHz. Un microprocesseur regroupe la plupart des composants de calcul sur un seul circuit. Il réalisait 60 000 opérations par seconde. Depuis, la puissance des microprocesseurs augmente exponentiellement. Le 8080 sorti en avril 1974 est encore plus puissant.Quels sont donc ces petits morceaux de silicium qui dirigent nos ordinateurs ?
}

\paragraph{
  Ces plus petit microprocesseur vont permettre de réduire significativement la taille des computers et d'en faire un produit grand public dans les années 70. L'industrie de l'informatique va passer du BTB au BTC.
}

\paragraph{
  En janvier 1975 sort l'Altair 8800. Développé par des amateurs frustrés par la faible puissance et le peu de flexibilité des quelques ordinateurs en kit existant sur le marché à l'époque, de fut certainement le premier ordinateur personnel en kit produit en masse. Il était le premier ordinateur à utiliser un processeur Intel 8080.
}

\paragraph{
  En 1975 sortira aussi l'IBM 5100, machine totalement intégrée avec son clavier et son écran, qui se contente d'une prise de courant pour fonctionner.
}

\paragraph{
  Le Homebrew Computer Club est un club d'informatique de la Silicon Valley entre 1975 et 1986. Le club est initié par Gordon French et Fred Moore, qui s’étaient rencontrés à Menlo Park et souhaitent alors rendre l’informatique plus facilement accessible au grand public. L'Altair 8800 devint le sujet princpal de la première réunion du club. C'est lors de la présentation de l'Altaire que Wozniak eut l'idée d'assembler un ordinateur personnel grace au microprocesseur.
}

\paragraph{
  En janvier 1975, la revue Popular Mechanics faisait sa couverture avec l'Altair, le premier micor-ordinateur en kit. Pour 195 dollars on pouvait avoir un tas de composants à souder soi-même et fabriquer un computer. Mais une fois monté l'ordinateur ne pouvait pas faire grand chose avec. Bill Gates et Paul Allen lurent le magazine et se mirent à écrire un BASIC pour l'Altair. La machine retint aussi l'attention de Steve Jobs et Steve Wozniak.
}

\paragraph{
  Les microprocesseurs Intel étant trop cher à l'époque et Wozniak se rabat donc sur un Motorola 6800. Il écrit même le code source sur papier comme il ne peut pas se payer l'utilisation d'un ordinateur! Le 29 juin 1975 la machine est prête. "C'est la première fois dans l'histoire que quelqu'un tapait un caractère sur un clavier et le voyait s'afficher sur l'écran de son ordinateur, juste sous ses yeux".
}

\paragraph{
  L'Apple I se vendit à 666 dollars et environ 200 machines.
}

\paragraph{
  L'Apple II sort en 1977. Malgré son prix élevé (environ 1000 dollars) il prend l'avantage sur le TRS-80 et le Commodore PET lancé la même année. C'est un symbole d'ordinateur personnel à l'époque.
}

\paragraph{
  Steve Jobs et Bill Gates ne voulaient pas partager leur découvertes au *Homebrew Computer Club*. Une fois débarassé du partager leurs connaissances au club ils fondèrent leur société. Jobs convincut Wozniak de démissioner pour monter Apple. Et Gates a fondé Microsoft pour finaliser les ventes de BASIC pour l'Altair. Les libristes firent leur retour bien des années plus tard.
}

\paragraph{
  Steve Jobs pressentait un gros marché s'il arrivent à démocratiser ces machines, à les rendre plus intuitive, ergonomique. Et passer d'un marché BtoB à un marché BtoC avec des millions de nouveaux clients potentiels. Ils travaillaient donc d'arrache pied sur un système d'exploitation avec un environnement graphique au lieu d'un interprétateur de ligne de commandes.
}

\paragraph{
  Alan Kay chercheur au Xerox Parc travaille à l'époque sur des interfaces conviviales susceptibles de remplacer les lignes de commandes qui intimidaient tant le profane. Cette interface graphique était rendue possible par le bitmapping, soit l'affichage d'images matricielles. Chaque pixel est piloté point par point par la mémoire de l'ordinateur qui lui indique si le pixel est éteint ou allumé. L'image matricielle et les interfaces graphiques devinrent la caractéristiques des ordinateurs du PARC.
}

\paragraph{
  Xerox alors investisseur d'Apple autorisa Steve Jobs à rendre visite au centre de recherche situé à Palo Alto. À la différence des dirigeants de Xerox il sur tout de suite l'importance de ses travaux de recherche. S'agit-il d'un vol d'Apple ou d'une bourde de Xerox? Un peu des deux.
}


\chapter{ 1970 }

\input{book/chap07}

\chapter{ 1980 : L'explosion de l'ordinateur personnel }

\paragraph{
  Dans les années 80, plusieurs innovations techniques vont jouer une rôle majeur pour démocratiser les ordinateurs et passer du computer au personnal computer.
}

\paragraph{
  D'abord, les composants des ordinateurs (processeurs, etc.) deviennent de plus en plus puissant année après année. Il traite toujours plus de calculs avec un prix de moins en moins cher.
}

\section*{Les insolubles problèmes de compatibilité}

\paragraph{
  C'est en 1981 qu'IBM lance son premier ordinateur personnel, l'IBM PC. Il fonctionne avec un terminal classique, c'est à dire uniquement avec un invite de commande. Apple, la société de Steve Jobs et Steve Wozniak lui emboîte le pas avec l'Apple I et le Lisa deux ans plus tard. La limite de ces ordinateurs est l'incompatbilité des leurs logiciels. Il est fréquent que les programmes ne soient pas compatibles même entre ordinateurs d'une même marque. C'est à dire qu'un logiciel écrit pour l'Apple I ne fonctionnera pas pour le Lisa. L'ergonomie est très primaire. Tout se fait par des lignes de commande qui se lance sur un terminal tout de noir vêtu.
}

\section*{Microsoft}

\paragraph{
  Comme vendeur de logiciels le plus gors est à l'époque Microsoft. La fime de Seattle avait écrit quelques application pour l'Apple II dont le tableau Multiplan, futur Excel.
}

\paragraph{
  Lancé en 1975, la société Microsoft est une des premières à s'être spécialisée dans les logiciels. Gates avait fondé Microsoft pour finaliser la vente de leur Basic pour l'Altair. Basic est un langage de programmation conçu à l'origine pour permettre à des néophytes en informatique d'écrire des programmes pouvant tourner sur divers ordinateurs. A l'époque le système d'exploitation (OS) pour les IBM est le MS-DOS (PC-DOS). Il fonctionne avec des invites de commandes. Microsoft s'est réservé le droit de commercialiser sa propre version du MS-DOS à des ordinateurs non-IBM. IBM accepte en échange d'un ristourne sur le prix de la licence MS-DOS. Il ne le sait pas encore mais IBM vient de signer son propre arrêt de mort sur le marché des PC.
}

\paragraph{
  En effet à part Apple qui choisit de construire son propre OS, les nouvelles sociétés de Hardware vont utiliser Microsoft. Et une va particulièrement tirer son épingle du jeu: Compaq.
}

\section*{Un nouvel arrivant: Compaq}

\paragraph{
  Dans les années 80 après la sortie des premiers ordinateurs personnels destinés aux grand public, Compaq invente l'ordinateur portable. La compagnie originaire de Houston sort un ordinateur de la taille d'une malette équipé d'une poignée. Créé au début des années 1980, la société Texanne a été parmi les premières à prendre au sérieux les problèmes de compatibilité. Pour concurencer IBM qui regnait en maître sur le marché, Compaq a mis en place un compatibilité entre son ordinateur et certains logiciels tournant sur IBM.
}

\paragraph{
  Par exemple à l'époque, une société qui achetait 100 ordinateurs Apple II puis faisait développer des logiciels, en général, par un société externe. Si ensuite elle souhaitait s'aggrandir et acheter des nouveaux ordinateurs d'une autre marque, tout le software était à refaire. En faisant du *retro-ingeniering* Compaq vendait des PC compatibles. Mais seulement avec des IBM et pas des Apple. Devant le succès commercial du Compaq, IBM décida de sortir lui aussi son portable (lequel??). Mais les *softwares* des autres IBM n'étaient pas comptabibles avec le PC portable d'IBM. En résumé si vous aviez développé des logiciels sur un IBM puis que vous souhaitiez renouvellliez votre stock de vieux PC, les logiciels étaient bon pour la poubelle.
}

\section*{Le MacIntosh}

\paragraph{
  En 1983 Apple sort le Lisa. C'est la premier ordinateur avec une interface graphique. Son échec commercial fut cuisant mais Apple réitère en 1984 avec le MacIntosh. Steve Jobs est convaincu de la pertinence d'une interface graphique pour toucher un public plus large que les spécialistes et équiper tous les foyers américains d'un ordinateur personnel. C'est au cours d'une visite dans les locaux de XeroX que Steve Jobs trouve l'inspiration pour ses deux innovations majeurs.
}

\paragraph{
  La première innovation est la fenêtre graphique (*window*) en tant que répresentation visuelle d'un dossier (*directory*). La deuxième est la souris qui permettait une utilisation facile des fichiers. Les deux ne sont pas breveté par XeroX. Steve Jobs est le premier a avoir l'idée d'utilisation la souris pour la manipulation des fichiers (*files*) et dossiers (*directory*).
}

\paragraph{
  Le premier MacIntosh (1984) était né. C'est le premier PC d'Apple même s'il n'est pas encore transportable. C'est le premier PC de l'histoire avec une souris et des fenêtres de navigation sur le bureau. Le MacInstosh était très lent. Il n'avait que 128kO de RAM et un lecteur de disquette faisant office de disque dur. C'est à dire pas de mémoire. Les logiciels n'étaient pas compatibles. Il reprenait plusieurs caractéristiques du Lisa, comme le processeur Motorola 68000, mais pour un prix bien plus abordable 2 500 dollars. Ils ont pu obtenir ce prix bas grâce à l'abandon de quelques fonctionnalités comme le multitâche. Plusieurs applications utilisait la souris comme MacPaint et MacWrite.
}

\paragraph{
  Malheureusement Steve Jobs ne survivra pas à l'échec du Lisa puisqu'il fut viré de l'entreprise en septembre 1985. Apple ne sortira plus rien d'innovant après cela. Un retard fatal, l'industrie va avancer sans elle et Microsoft va imposer ses normes. Apple se retrouva hors-jeu pour un long moment.
}

\paragraph{
  C'est le 27 septembre 1983 que commence le projet GNU. Le projet fondé par Richard Stallman a l'ambition de créer des software non protégés pas des brevets et surtout compatible. La compatiblité pouvant être vu comme un implication d'un software libre. En effet si le code est libre il est plus facile pour les créateurs de hardware de fabriquer une machine qui lit le logiciel. En plus de logiciels GNU veut aussi sortir un OS pouvant tourner sur plusieurs machines. C'est bien après la bataille, seulement en 1997, que GNU sort une interface graphique pour son OS.
}

\section*{1985: Windows lance sur *Operating System*}

\paragraph{
  En 1985, Windows sort son premier OS Windows 1.0. Une version médiocre et un échec commercial dû en partie à sa mauvaises gestion des fenêtres. Elles n'avaient pas de bouton de fermeture et elle ne se chevauchait pas. Le nouveau CEO d'Apple, Sculley, voulait poursuivre en justice Microsoft pour avoir copier le système des fenêtres du MacIntosh. Mais Microsoft menaca de bloquer les ventes de logiciels pour Mac (Word et Excel). À l'époque, Bill Gates déclare: "Xerox était notre riche voisin à tous les deux, et que je suis entré chez lui pour lui voler sa télévision, j'ai découvert que [Steve Jobs] l'avait déjà emporté".
}

\section*{La parenthèse NeXt}

\paragraph{
  En 1988, Steve Jobs fonde NeXT, une société d'ordinateur pour les centres de recherches. Le système d'exploitation s'appelait NeXTStep. IBM l'utilise un temps pour mettre en concurrence Windows. Compaq et Dell voulait aussi l'iutiliser. Steve Jobs investit dans Pixar, le département informatique de LucasFilm. Elle devient alors une société indépendante. En déconfiture à l'époque, Pixar gérait le graphisme et les animations 3D que Steve vouait intégrer au NeXt. En 1990, NeXT arrête de fabriquer des ordinateurs pour se concentrer sur l'OS.
}

\paragraph{
  Dans son autobiographie, Richard Stallman analyse ce changement de paradigme sur le marché informatique: "Les logiciels autrefois considéré comme une forme de garniture offerte par les fabricants pour donner plus saveurs à leurs coûteux systèmes informatiques, deviennent rapidement le plat principal."
}

\paragraph{
  La fin des années 80 marque la prise de pouvoir du software sur le hardware, c'est à dire des logiciels sur la machine. Et justement, le logiciel qui va devenir le plat principal est en train de naître.
}

\section*{1989: La naissance du World Wide Web}

\paragraph{
  C'est en 1989 que le chercheur britannique Tim Berners-Lee invente le World Wide Web sur une machine NeXT. Il travaille au CERN un laboratoire de physique des particules.
}

\paragraph{
  *À l’origine, le projet, baptisé « World Wide Web », a été conçu et développé pour que des scientifiques travaillant dans des universités et instituts du monde entier puissent s'échanger des informations instantanément.*
}

\paragraph{
  *L'idée de base du WWW était de combiner les technologies des ordinateurs personnels, des réseaux informatiques et de l'hypertexte pour créer un système d'information mondial, puissant et facile à utiliser.* par la communauté des chercheurs (site web du CERN)
}

\paragraph{
  *Le premier site Web créé au CERN, et dans le monde, était destiné au projet World Wide Web lui-même. Il était hébergé sur l’ordinateur NeXT de Tim Berners-Lee. En 2013, le CERN a entrepris de remettre en service ce premier site Web : info.cern.ch.* qui est aujourd'hui le site web du CERN. (site web du CERN)
}

\paragraph{
  Pour mieux comprendre ce que cela signifie, il faut s'intéresser au fonctionnement d'Internet: pour voir une page web, l'ordinateur reçoit un petit paquet de données pour l'image, un autre pour le texte, etc. Ces paquets de données voyagent des serveurs jusqu'à votre PC grâce à une infrastructure de câbles et de fibres, un peu comme des camions voyagent de Paris à Toulouse en passant par des autoroutes. C'est ce qu'on appelle l'internet. Sauf que pour faire cette route, il faut s'acquitter d'un droit de passage à l'entreprise qui possède les infrastructures. Sur Internet, chacun paie donc à des opérateurs de télécommunications (en France Orange, Bouygues, Free...) une certaine somme pour accéder à leur infrastructure, un peu comme un camion paie Vinci pour son trajet d'autoroute.
}

\paragraph{
  *Tim Berners-Lee écrit la première proposition de création du World Wide Web en mars 1989 et sa seconde proposition en mai 1990. Puis, en novembre 1990, l’ingénieur en systèmes belge Robert Cailliau le rejoint et ils élaborent ensemble une proposition formelle pour un système de gestion de l'information esquissant les concepts fondamentaux et définissant les principaux termes liés au Web. Le document décrit un « projet hypertexte » appelé WorldWideWeb, dans lequel un « web » (une toile) de « documents hypertextes » peut être vu par des « navigateurs ».*
}

\paragraph{
  *Fin 1990, Tim Berners-Lee rend opérationnel le premier serveur et navigateur Web au CERN, concrétisant ainsi ses idées. Il avait développé le code pour le premier serveur Web sur un ordinateur NeXT. Pour éviter qu'on ne l'éteigne accidentellement, une étiquette avait été collée sur l'ordinateur, où il était écrit à la main, en rouge : « Cette machine est un serveur. NE PAS ÉTEINDRE !! »*
}

\paragraph{
  *Info.cern.ch était l’adresse du tout premier site et serveur Web, qui était hébergé sur un ordinateur NeXT du CERN. L’adresse de la première page Web était http://info.cern.ch/hypertext/WWW/TheProject.html*
}

\paragraph{
  *La page comportait essentiellement des informations relatives au projet WWW, notamment une description de ce qu'est l'hypertexte, des détails techniques pour la création d'un serveur Web, et des liens vers d'autres serveurs, qui étaient ajoutés au fur et à mesure qu'ils devenaient disponibles.*
}

\paragraph{
  *La conception du WWW permettait d'avoir facilement accès à l'information existante ; une page Web rudimentaire proposait des liens utiles pour les scientifiques du CERN (par exemple l'annuaire du CERN, ou encore des guides d'utilisation des ordinateurs centraux du CERN). La recherche se faisait par mots-clés, car il n'y avait pas encore de moteurs de recherche.*
}

\paragraph{
  Le premier navigateur de Tim Berners-Lee fonctionnait uniquement sur NeXT. Le chercheur conçoit rapidement un navigateur pouvant être exécuté sur d'autres ordinateurs que des NeXT. Mais le navigateur était très rudimentaire et peu convivial même s'il permettait déjà de modifier les pages directement depuis le navigateur.
}

\paragraph{
  *En 1991, Tim Berners-Lee lance son premier logiciel WWW, qui incluait le navigateur en mode ligne, un logiciel pour le serveur Web et une bibliothèque pour les développeurs. En mars de cette même année, le logiciel devient accessible à d'autres collègues sur des ordinateurs du CERN.*
}

\paragraph{
  *Grâce aux efforts de Paul Kunz et Louise Addis, le premier serveur Web aux États-Unis est mis en ligne en décembre 1991, là aussi dans un laboratoire de physique des particules : le Centre de l'accélérateur linéaire de Stanford (SLAC), en Californie. Il n'y avait alors pour ainsi dire que deux sortes de navigateur. L'un était la version qui avait servi au développement initial, sophistiquée mais disponible uniquement sur des machines NeXT. L'autre était le navigateur en mode ligne, très simple à installer et à exécuter sur n'importe quelle plateforme, mais limité en puissance et peu convivial. Il était évident que la petite équipe du CERN à l'origine du système ne pourrait à elle seule effectuer le travail nécessaire pour le développer. Aussi Tim Berners-Lee lance-t-il un appel via l'Internet pour que d'autres développeurs viennent leur prêter main-forte. Plusieurs personnes créent alors des navigateurs, la plupart exécutables dans l'environnement X-Window. Les plus connus de cette époque sont MIDAS, de Tony Johnson (qui travaille alors auprès du SLAC), Viola, de Pei Wei (qui travaillait pour l'éditeur d'ouvrages techniques O'Reilly), et Erwise, d'un groupe d'étudiants finlandais de l'Université de technologie d'Helsinki.* De plus Emacs se développe aussi pour lire des pages web.
}


\chapter{ 1990 }

\paragraph{
  PRISM: En 2015, un ancien ingénieur de la NSA révèle au monde la surveillance qu'exerce les services secrets américains sur Internet. Le programme ultra secret répondant au nom de PRISM interceptent toutes les metadonnées des communications mondiales et les stockent dans des serveurs de la NSA nommé MDR pour Mission Data Repository.
}

\paragraph{
  Les entreprises américaines sont contraintes par le Patriot Act de céder leurs données.
}

\paragraph{
  HTTPS: Les entreprises de la planète on changé de plateforme pour leur site en remplaçant le http par le https qui contribue a empêcher une tierce partie d'intercepter le trafic web. La sécurité des informations transmises par le protocole HTTPS est basée sur l'utilisation d'un algorithme de chiffrement, et sur la reconnaissance de validité du certificat d'authentification du site visité.
}


\tableofcontents

\end{document}
