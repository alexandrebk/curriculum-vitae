
\documentclass[french, 11pt,a4paper]{moderncv}
\moderncvtheme[blue]{classic}
\usepackage[utf8]{inputenc}
\usepackage[top=1.1cm, bottom=1.1cm, left=1.5cm, right=1.5cm]{geometry}
% Largeur de la colonne pour les dates
\setlength{\hintscolumnwidth}{2.4cm}


\firstname{Alexandre}
\familyname{Bouvier}
\title{Fullstack Developer Ruby on Rails}
\address{17 rue Jamen Grand}{69300 Caluire et Cuire}
\email{alexandre.b2506@gmail.com}
\homepage{www.alexandrebouvier.fr}
\mobile{06 52 14 36 50}
\extrainfo{37 ans -- Permis B}
\begin{document}
\maketitle

\section{Expériences professionnelles}

  \cventry{Janvier 2023\\à Aujourd'hui}{FullStack Web Developper}{The Packengers}{Paris}{}{
    Ruby on Rails, PostGreSQL, MiniTest, Stimulus, Turbo, Github, Heroku, Notion \newline{}
    \begin{itemize}%
    \item Passage de l'api Ocotbat à Sellsy
    \newline{}
    \end{itemize}
  }

  \cventry{Janvier 2022\\à Décembre 2023}{FullStack Web Developper}{Freelance}{Paris}{}{
    Ruby on Rails, RSpec, Stimulus, Github, Heroku, Scalingo, Notion \newline{}
    API utilisé: Salesforce, Twillio, Airtable, Pipedrive
    \begin{itemize}%
    \item Maintenance www.mon-cdi.fr, www.studio-paillette.fr
    \item Création www.kinship.fr
    \item Développement d'une api web pour Vera App
    \newline{}
    \end{itemize}
  }

  \cventry{Avril 2019\\à Déc. 2021}{FullStack Web Developper}{Peers And Peers}{Paris}{}{
    Ruby on Rails, RSpec, Stimulus, GitLab, Heroku, Pivotal Tracker, Metabase
    Funnel de conversion, Devis, Facturation \newline{}
    API utilisé: Yousign, Twillio, Slack,
    \begin{itemize}%
    \item Développement d'une application web pour Les Ateliers NX
    \item Développement d'une application web pour Monsieur Peinture
    \newline{}
    \end{itemize}
  }

  \cventry{Avril 2018\\à Déc. 2023}{Teacher et Teacher Assistant}{Le Wagon}{France Suisse et Allemagne}{Français et Anglais}{
    Enseignement et soutien aux élèves sur le parcours Full Stack Web Developper (Ruby on Rails)
  }

  \cventry{Août 2016\\à Déc. 2017}{Import Manager}{French Feast}{New York et Paris}{}{
    \begin{itemize}%
    \item Mise en place d'un ERP (Odoo) pour la gestion de l'entrepôt
    \item Gestion du parc des PC
    \item Gestion du site web (B2C)
    \newline{}
    \end{itemize}
  }

  \cventry{Janvier 2015\\à  Juil. 2016}{Opérations et Logistiques}{Uber}{Paris}{}{
    \begin{itemize}%
    \item Lancement de l'offre UberEats à Paris (première ville hors US)
    \item Gestion opérationnelle et recrutement des livreurs
    \item Formation des livreurs - https://youtu.be/G1fgFAX6vCE
    \newline{}
    \end{itemize}
  }

  \cventry{Janvier 2013\\à  Déc. 2014}{Responsable E-commerce}{Aux-concours.com}{Paris}{}{
    \begin{itemize}%
    \item Gestion du site (sur Joomla)
    \item Création et lancement de nouvelles offres
    \newline{}
    \end{itemize}
  }

  \cventry{Janvier 2012\\à  Déc. 2012}{Chargé de cours en Mathématiques}{Université Paris I}{Paris}{}{Chargé de cours en Mathématiques pour les étudiants en Licence 1 d'Economie\newline{}}

\section{Formations}

  \cventry{Janvier\\à Mars 2018}{Développeur web full stack Ruby}{Le Wagon}{Batch 121}{}{Bootcamp intensif pour se former à HTML, CSS, Bootstrap, JavaScript, jQuery, SQL, Git, GitHub, Heroku et Ruby on Rails. Design, implémentation et production d'un clone AirBnb et d'un projet personnel avec Rails}

  \cventry{2008 -- 2011}{Master Professionnel}{Université Panthéon-Sorbonne Paris I}{}{}{Gestion et Méthodes de Décision d’Entreprise}

  \cventry{2005 -- 2008}{Licence Économie et Statistiques}{Université Panthéon-Sorbonne Paris I}{}{}{Option économétrie}

  \cventry{2005}{Baccalaureat}{Lycée Racine, Paris}{}{}{Série scientifique, spécialité Physique-Chimie}

% \section{Compétences}
  % \cvitem{Langages}{Ruby, Javascript, HTML, CSS}
  % \cvitem{Framework}{Ruby on Rails, Stimulus, jQuery}
  % \cvitem{BDD}{MySQL, PostGreSQL}
  % \cvitem{CMS}{Joomla, Moodle, Odoo}
  % \cvitem{OS}{MacOS, Linux (Debian, Ubuntu)}
  % \cvitem{Administration}{bash, git, Apache2, BIND}
  % \cvlanguage{Anglais}{lu, parlé, écrit}{}

\end{document}\grid
