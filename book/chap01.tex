\paragraph{
  Il y'a eu une première révolution industrielle ou on avait inventé le moteur. Et ce moteur a remplacé le muscle humain c'est à dire qu'on a remplacé des ouvriers qui utilisait leur muscles par des moteurs. Parce que c'est moins chère. Ca a mécanisé le travail agricole et le travail manufacturier. Mais ça l'a pas totalement remplacé.
}

\paragraph{
  L'invention de l'ordinateur c'est la révolution industrielle de la dématérialisation du cerveau humain. Cad que les comptables qui faisait des calculs avec le grand livre. Maintenant le grand livre est stocke dans un ordinateur et le calcul est lui aussi fait par l'ordinateur. Et les comptables peuvent se concentrer sur l'optimisation fiscale. On pourrait imaginer que plus tard le *machine learning* remplace l'optimisation fiscale mais ce n'est pas un roman d'anticipation. Y'a plein de job qui se font automatisé. L'informatique c'est la machine outil pour faire remplacer les métiers de service.
}

\paragraph{
  Basile Bouchon en 1725 est le premier à utilisé une programmation à ruban perforé pour les métiers à tisser.
}

\paragraph{
  Tout commence avec le recensement aux États-Unis en 1890. Un concours est lancé pour trouver une méthode rapide pour compter les Américains. Herman Hollerith propose un tabulateur. Le comptage est terminé en six semaines avec cinq millions de dollars d'économie. La machine sort: 62 927 766 ressortissants. Le calculateur analogique qui compte le nombre de trous percés dans chaque carte.
}

\paragraph{
  Il s'agissait d'une machine à cartes perforées. Les trous déclenchent la progression des compteurs. Un prémisse de l'ordinateur. Babbage avait eu la même idée avant lui auparavant mais Hollerith en a fait un business: International Business Machines Corporation (IBM). Et il dépose un brevet de la machine à cartes perforées.
}

\paragraph{
  En 1939, les nazis souhaitent aussi faire un recensement de la population et font appel à Dehomag, une succursale allemande d'IBM.
}
