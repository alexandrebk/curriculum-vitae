\section{
  Ordinateurs analogiques.
}

\paragraph{
  Avant la seconde guerre mondiale, les ordinateurs analogiques étaient tous mécaniques ou électriques. Ces ordinateurs utilisaient des quantités physiques, telles que la tension,le courant ou la vitesse de rotation des axes pour représenter les nombres. À partir de la seconde guerre mondiale, le calcul numérique remplaça le calcul analogique.
}

\paragraph{
  Les premières machine de Turing ou plutôt machine Turing-compatible servent à calculer des trajectoires de missiles. Cela permettait de calculer des trajectoires de tirs beaucoup plus rapidement qu'un être humain le ferait. Et en temps de guerre, les secondes sont précieuses. En 1938, Konrad Zuse commença la construction des premières séries-Z, des calculateurs électromécaniques comportant une mémoire et une programmation limitée. La Wehrmacht utilisa ces systèmes pour des missiles guidés.
}

\paragraph{
  Durant la même période., en 1938, John Vincent Atanasoff et Clifford E. Berry, de l'Université de l'État de l'Iowa, développèrent l'ordinateur Atanasoff-Berry un additionneur à 16 bits. Cette machine avait pour but de résoudre des systèmes d'équation linéaires. La mémoire était stcokée à l'aide de condensateurs fixés à un tambour rotatif.
}

\paragraph{
  Les trois idées propres aux ordinateurs d'après guerre sont:
  - l'utilisation du sytème binaire (plus fiable et plus simple à mettre au point que le système décimal)
  - la séparation entre le calcul et la mémoire (Von Neumann ?)
  - l'utilisation de composants électroniques plutôt que des éléments mécaniques pour réaliser les calculs
}
