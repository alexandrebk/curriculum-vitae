\paragraph{
  En 1952, IBM produit son premier ordinateur scientifique, l'IBM 701. Il est destiné à la défense américaine. Il effectuait 16 000 additions ou 2 200 multiplications par seconde. L'objectif était de calculer des trajectoires de missiles.
}

\paragraph{
  En 1953, IBM lance l'IBM 650, ordinateur commercial (ou de gestion). Il se présentait en 2 modules de 2,5 m3, l'un de 900kg contenant l'ordinateur, l'autre de 1350kg contenant son alimentation électrique.
}

\paragraph{
  Le premier langage de programmation universel de haut niveau à être implémenté, le Fortran (Formula Translator), fut aussi développé par IBM à cette période.
}

\paragraph{
  En 1958, IBM rate l'opportunité de racheter une jeune société en expansion qui avait mis au point une nouvelle technologie d'impressions appelée la xerographie. Deux ans plus tard va naître Xerox.
}

\paragraph{
  En 1960, la Compagnie des Machines Bull (France) sort le Gamma 60. C'est le premier ordinateur multitâches dans le monde et l'un des premiers à comporter plusieurs processeurs (voir multiprocesseur).
}

\paragraph{
  La même année, Digital Equipment Corporation (DEC) lança le PDP-1 (Programmed Data Processor). Le PDP-1 était le premier ordinateur interactif et a lancé le concept de mini-ordinateur. Il effectuait 100 000 opérations par seconde. Vendu pour 120 000 dollars.
}
